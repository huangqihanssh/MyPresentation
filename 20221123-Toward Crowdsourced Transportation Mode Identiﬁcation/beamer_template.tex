%%%%%%%%%%%%%%%%%%%%%%%%%%%%%%%%%%%%%%%%%
% Beamer Presentation
% LaTeX Template
% Version 1.0 (10/11/12)
%
% This template has been downloaded from:
% http://www.LaTeXTemplates.com
%
% License:
% CC BY-NC-SA 3.0 (http://creativecommons.org/licenses/by-nc-sa/3.0/)
%
%%%%%%%%%%%%%%%%%%%%%%%%%%%%%%%%%%%%%%%%%

%----------------------------------------------------------------------------------------
%	PACKAGES AND THEMES
%----------------------------------------------------------------------------------------

\documentclass{beamer}
\setbeamertemplate{caption}[numbered]
\setbeamertemplate{bibliography item}[text]
\mode<presentation> {

% The Beamer class comes with a number of default slide themes which change the colors and layouts of slides. Below this is a list of all the themes, uncomment each in turn to see what they look like.
%
%\usetheme{default}
%\usetheme{AnnArbor}
%\usetheme{Antibes}
%\usetheme{Bergen}
%\usetheme{Berkeley}
%\usetheme{Berlin}
%\usetheme{Boadilla}
%\usetheme{CambridgeUS}
%\usetheme{Copenhagen}
%\usetheme{Darmstadt}
%\usetheme{Dresden}
%\usetheme{Frankfurt}
%\usetheme{Goettingen}
%\usetheme{Hannover}
%\usetheme{Ilmenau}
%\usetheme{JuanLesPins}
%\usetheme{Luebeck}
\usetheme{Madrid}
%\usetheme{Malmoe}
%\usetheme{Marburg}
%\usetheme{Montpellier}
%\usetheme{PaloAlto}
%\usetheme{Pittsburgh}
%\usetheme{Rochester}
%\usetheme{Singapore}
%\usetheme{Szeged}
%\usetheme{Warsaw}

% As well as themes, the Beamer class has a number of color themes
% for any slide theme. Uncomment each of these in turn to see how it
% changes the colors of your current slide theme.

%\usecolortheme{albatross}
%\usecolortheme{beaver}
%\usecolortheme{beetle}
%\usecolortheme{crane}
%\usecolortheme{dolphin}
%\usecolortheme{dove}
%\usecolortheme{fly}
%\usecolortheme{lily}
%\usecolortheme{orchid}
%\usecolortheme{rose}
%\usecolortheme{seagull}
%\usecolortheme{seahorse}
%\usecolortheme{whale}
%\usecolortheme{wolverine}

%\setbeamertemplate{footline} % To remove the footer line in all slides uncomment this line
%\setbeamertemplate{footline}[page number] % To replace the footer line in all slides with a simple slide count uncomment this line

%\setbeamertemplate{navigation symbols}{} % To remove the navigation symbols from the bottom of all slides uncomment this line
}
%\usepackage{bookman} % the used font
\usepackage{graphicx} % Allows including images
\usepackage{booktabs} % Allows the use of \toprule, \midrule and \bottomrule in tables

%----------------------------------------------------------------------------------------
%	TITLE PAGE
%----------------------------------------------------------------------------------------
\title[About Beamer] %optional
{About the Beamer class in presentation making}

\subtitle{A short story}

\author[Arthur, Doe] % (optional, for multiple authors)
{A.~B.~Arthur\inst{1} \and J.~Doe\inst{2}}

\institute[VFU] % (optional)
{
  \inst{1}%
  Faculty of Physics\\
  Very Famous University
  \and
  \inst{2}%
  Faculty of Chemistry\\
  Very Famous University
}

\date[VLC 2020] % (optional)
{Very Large Conference, April 2013}

%\logo{\includegraphics[height=0.5cm]{logo.png}}

%----------------------------------------------------------------------------------------
%	Highlight the title of the current section
%----------------------------------------------------------------------------------------
\AtBeginSection[]
{
  \begin{frame}
    \frametitle{Table of Contents}
    \tableofcontents[currentsection]
  \end{frame}
}



%----------------------------------------------------------------------------------------
%	All pages in the slides
%----------------------------------------------------------------------------------------
\begin{document}

% insert title page---------------------------
\frame{\titlepage}

%insert contents------------------------------
\begin{frame}
\frametitle{Table of Contents}
\tableofcontents
\end{frame}

\section{Introduction}

% insert a sample frame without animation--------------------------------
\begin{frame}
\frametitle{Sample frame title without animation}

This is a text in second frame. For the sake of showing an example.

\begin{itemize}
 \item Text visible on slide 1
 \item Text visible on slide 2
 \item Text visible on slide 3 % pay attention to this line, you can omit the content by omit the '-'
 \item Text visible on slide 4
\end{itemize}

\end{frame}

% insert a sample frame with animation 1 ----------------------------------
\begin{frame}
\frametitle{Sample frame title with animation}
This is a text in second frame. For the sake of showing an example.

\begin{itemize}
 \item<1-> Text visible on slide 1
 \item<2-> Text visible on slide 2
 \item<3> Text visible on slide 3 % pay attention to this line, you can omit the content by omit the '-'
 \item<4-> Text visible on slide 4
\end{itemize}

\end{frame}


% insert a sample frame with animation 2 -----------------------------------
\begin{frame}
 In this slide \pause

 the text will be partially visible \pause

 And finally everything will be there
\end{frame}



\section{Related work}

% insert a sample frame with highlighe -----------------------------------
\begin{frame}
\frametitle{Sample frame title}

In this slide, some important text will be
\alert{highlighted} because it's important.
Please, don't abuse it.

\begin{block}{Remark}
Sample text
\end{block}

\begin{alertblock}{Important theorem}
Sample text in red box
\end{alertblock}

\begin{examples}
Sample text in green box. The title of the block is ``Examples".
\end{examples}
\end{frame}

\section{Proposed method}

% insert a sample frame with formula -----------------------------------
\begin{frame}
\frametitle{Sample frame with formula}

In this slide, we insert an equation (\ref{E1}), which is the definition of $y$.

\begin{equation}
\label{E1}
y=
\begin{cases}
\frac{1}{f(x)},\ f(x) \neq 0 \\
f(x), \ f(x)=0
\end{cases}
\end{equation}

\end{frame}


\section{Experiments}

% insert a sample frame with a figure -----------------------------------
\begin{frame}
\frametitle{Dataset}

The dataset information is shown in Fig. \ref{dataset}.
%
%\begin{figure}[ht]
%\centering
%\includegraphics[width=0.8\textwidth]{dataset.png}
%\caption{A picture downloaded from the Internet.png} 
%\label{dataset}
%\end{figure}

\end{frame}

% insert a sample frame with a table -----------------------------------
\begin{frame}
\frametitle{A sample frame with a table}

\begin{table}[ht]
\caption{A sample table}
\label{T1}
\centering
\begin{tabular}{p{1.5cm}|p{1.5cm}|p{1.5cm}}
\hline
Metrics   &   M1 &	M2	\\
\hline
Accuracy  &  87\% &	88\%	\\
Precision &  91\% &	90\%	\\
Recall	  & 75\% &	77\%	\\
\hline
\end{tabular}
\end{table}

\end{frame}

\section{Conclusion}

\begin{frame}
\frametitle{Two-column slide}

\begin{columns}

% insert a sample frame with two columns --------------------------------

\column{0.5\textwidth}
This is a text in first column.
$$E=mc^2$$
\begin{itemize}
\item First item
\item Second item
\end{itemize}

\column{0.5\textwidth}
This text will be in the second column
and on a second tought this is a nice looking
layout in some cases.
\end{columns}
\end{frame}

% insert a sample frame with citation --------------------------------
\begin{frame}
\frametitle{Literature review}
\begin{itemize}
\item You can cite a paper like this \cite{team2015common,eiram2013cvssv2}.
\end{itemize}
\end{frame}


% insert a reference frame before the 'thank you' frame ----------------------
\begin{frame}
\frametitle{References}

\begin{thebibliography}{99} % Beamer does not support BibTeX so references must be inserted manually as below
\bibitem{team2015common}
Team, C.: Common vulnerability scoring system v3. 0: Specification document.
  First. org  (2015)
  
\bibitem{eiram2013cvssv2}
Eiram, C., Martin, B.: The cvssv2 shortcomings, faults, and failures
  formulation. In: Technical report, Forum of Incident Response and Security
  Teams (FIRST) (2013)
  
\end{thebibliography}
\end{frame}


% Insert a thank your frame ------------------------------------------------
\begin{frame}
\Huge{\centerline{Thank you!}}
\end{frame}

\end{document}
